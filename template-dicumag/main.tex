\documentclass[10pt]{article}
\usepackage[a4paper,margin=1in]{geometry}
\usepackage[utf8]{inputenc}
\usepackage[spanish]{babel}
\usepackage{graphicx}
\usepackage{xcolor}
\usepackage{hyperref}
\usepackage{fancyhdr}

% Rutas para los recursos gráficos (permite compilar desde la raíz o desde esta carpeta)
\graphicspath{{./}{./logo/}{./image/}}

% ============================================================
% CONFIGURACIÓN BÁSICA DEL TEMPLATE
% ============================================================
\newcommand{\institucion}{Universidad de Magallanes}
\newcommand{\facultad}{Departamento de Ingeniería en Computación}
\newcommand{\programa}{Ingeniería Civil en Computación e Informática}
\newcommand{\curso}{Taller de Integración}
\newcommand{\docente}{Dra. Patricia Maldonado}
\newcommand{\estudianteUno}{Diego Vidal}
\newcommand{\estudianteDos}{Bruno Martinez}
\newcommand{\titulo}{\textit{Aplicación de navegación interna para la Facultad de Ingeniería}}
\newcommand{\fechaentrega}{\textit{Fecha de entrega}}

\pagestyle{fancy}
\fancyhf{}
\fancyhead[L]{\institucion}
\fancyhead[R]{\curso}
\fancyfoot[C]{\thepage}
\setlength{\headheight}{14.5pt}

\begin{document}

% ============================================================
% PORTADA
% ============================================================
\begin{center}
  \begin{tabular}{@{}c@{\hspace{1cm}}c@{\hspace{1cm}}c@{}}
    \raisebox{-0.5\height}{\includegraphics[height=2.5cm]{umag.png}} &
    \raisebox{-0.5\height}{%
      \begin{minipage}[c]{6cm}
        \centering
        {\Large \curso}\\[2mm]
        {\large \institucion}\\[3mm]
        {\Large \titulo}
      \end{minipage}%
    } &
    \raisebox{-0.5\height}{\includegraphics[height=2.5cm]{dic.png}}
  \end{tabular}
\end{center}

\vspace{4mm}
\begin{center}
  \includegraphics[width=0.65\textwidth]{placeholder.png}
\end{center}

\vspace{4mm}
\hrule
\vspace{1mm}
\hrule

\vspace{3mm}
\begin{tabular}{ll}
  Estudiantes: & \estudianteUno\\
              & \estudianteDos\\
  Programa: & \programa\\
  Departamento: & \facultad\\
  Profesor adjunto: & \docente\\
  Fecha: & \fechaentrega\\
\end{tabular}

\vspace{3mm}
\hrule
\vspace{1mm}
\hrule

\clearpage
\tableofcontents

\clearpage

% ============================================================
% RESUMEN
% ============================================================
\section*{Resumen}
\textit{El presente proyecto desarrolla una aplicacion de navegacion interna que tiene como fin optimizar la orientacion y el desplazamiento dentro de la facultad de ingenieria de la universidad de magallanes. Esta solucion viene de la necesidad de mejorar la accebilidad y la eficiencia a la hora de encontrar salas, oficinas y laboratorios, especialmente para estudiantes nuevos y personas visitantes. Para el desarrollo se implemento un sistema de mapeo digital basados en mapas arquitectonicos del edificio, integrando tecnologias de posicionamiento basadas en puntos de referencia.}

% ============================================================
% INTRODUCCIÓN
% ============================================================
\section{Introducción}
% La introducción debe tener las siguientes subsecciones según el formato [cite: 40]

\subsection{Antecedentes}
% 
Los alumnos cuando llegan por primera vez a la universidad pueden sentirse desorientados a la hora de ir a alguna sala o laboratorio ya que no sabe como llegar a esos lugares en especifico, para resolver esta problemática se busca crear una aplicación móvil para poder navegar dentro de la facultad.

\subsection{Período en que se realizó}
% 
El proyecto fue desarrollado en el segundo semestre del año 2025 en las dependencias de la facultad de ingeniería de la universidad de Magallanes. Las pruebas y evaluaciones se realizaron usando planos digitales y simulaciones de desplazamiento en sus principales pasillos y áreas de uso común.

\subsection{Objetivo General}
% 
Desarrollar una aplicación móvil de navegación interna que permita a los usuarios ubicarse y desplazarse eficientemente dentro de la Facultad de Ingeniería, utilizando tecnologías de posicionamiento y mapas digitales interactivos.

\subsection{Objetivos Específicos}
% 
\begin{itemize}
  \item Analizar la estructura espacial y las necesidades de orientación dentro de la Facultad de Ingeniería.
  \item Diseñar e implementar una interfaz intuitiva que muestre mapas internos y rutas óptimas hacia los distintos destinos.
  \item Integrar tecnologías de posicionamiento en interiores (beacons, Wi-Fi o códigos QR) que permitan determinar la ubicación del usuario en tiempo real.
\end{itemize}

% ============================================================
% DESARROLLO / HALLAZGOS / RESULTADOS
% ============================================================
\section{Desarrollo/Hallazgos/Resultados}
% 
% Esta sección se divide en Metodología y Presentación de Resultados [cite: 61]

\subsection{Metodología utilizada}
% 
Para la realización de este proyecto se siguieron las siguientes etapas:

\begin{enumerate}
  \item Digitalización de planos arquitectónicos de la facultad.
  \item Investigación sobre la creación de aplicaciones móviles.
  \item Preparación del entorno de desarrollo y repositorio de Github.
  \item Creación de la interfaz gráfica de la aplicación.
  \item Integración de los mapas a la aplicación.
  \item Modelado del grafo de salas y conexiones.
  \item Implementación del algoritmo de rutas.
  \item Implementación de funcionalidades extras a la aplicación.
  \item Pruebas piloto de la aplicación.
  \item Escalado a todos los pisos.
  \item Pruebas finales y ajustes.
\end{enumerate}

\subsection{Presentación de los Hallazgos y Resultados}
% 
% Los títulos de esta sección deben coincidir con los objetivos específicos [cite: 69]
Incluya diagramas, tablas o fragmentos de código relevantes. Ejemplo de uso de figuras:

\begin{figure}[htbp]
  \centering
  \includegraphics[width=0.85\textwidth]{placeholder.png}
  \caption{Sustituir esta imagen por el diagrama o captura correspondiente}
\end{figure}

% ============================================================
% CONCLUSIONES Y RECOMENDACIONES
% ============================================================
\section{Conclusiones y Recomendaciones}
% 
Redacte conclusiones breves destacando los aprendizajes clave y si se cumplieron los objetivos[cite: 83]. Añada recomendaciones si es pertinente[cite: 86].

% ============================================================
% REFERENCIAS
% ============================================================
\section{Referencias}
% [cite: 33]
% El formato solicitado es IEEE.
% Asegurarse de que las citas en el texto usen corchetes, ej. [1], [2].
\begin{thebibliography}{9}
\bibitem{torvalds2001}
L. Torvalds y D. Diamond, \textit{Just for Fun: The Story of an Accidental Revolutionary}. HarperBusiness, 2001.

\bibitem{tanenbaum2015}
A. S. Tanenbaum y H. Bos, \textit{Modern Operating Systems}. 4.\,ed., Pearson, 2015.
\end{thebibliography}

% ============================================================
% ANEXOS
% ============================================================
\section{Anexos}
% [cite: 34, 87]
% Aquí se puede incluir material que complementa el informe pero que
% no es esencial para la comprensión principal[cite: 90, 91].

\vspace{0.5cm}
\hrule
\begin{center}
\textbf{\estudianteUno{} y \estudianteDos}\\
Estudiantes de \programa\\
\institucion\\
\fechaentrega{}
\end{center}

\end{document}