\documentclass[10pt]{article}
\usepackage[a4paper,margin=1in]{geometry}
\usepackage[utf8]{inputenc}
\usepackage[spanish]{babel}
\usepackage{graphicx}
\usepackage{xcolor}
\usepackage{hyperref}
\usepackage{fancyhdr}

% Rutas para los recursos gráficos (permite compilar desde la raíz o desde esta carpeta)
\graphicspath{{./}{./logo/}{./image/}}

% ============================================================
% CONFIGURACIÓN BÁSICA DEL TEMPLATE
% ============================================================
\newcommand{\institucion}{Universidad de Magallanes}
\newcommand{\facultad}{Departamento de Ingeniería en Computación}
\newcommand{\programa}{Ingeniería Civil en Computación e Informática}
\newcommand{\curso}{Clase de Universidad 101}
\newcommand{\docente}{Dra. María González}
\newcommand{\estudiante}{Juan Pérez}
\newcommand{\titulo}{\textit{Título provisional del informe}}
\newcommand{\fechaentrega}{\textit{Fecha de entrega}}

\pagestyle{fancy}
\fancyhf{}
\fancyhead[L]{\institucion}
\fancyhead[R]{\curso}
\fancyfoot[C]{\thepage}
\setlength{\headheight}{14.5pt}

\begin{document}

% ============================================================
% PORTADA
% ============================================================
\begin{center}
  \begin{tabular}{@{}c@{\hspace{2cm}}c@{\hspace{2cm}}c@{}}
  \includegraphics[height=2.8cm]{umag.png} &
    \begin{minipage}[c]{5.5cm}
      \centering
      {\Large \curso} \\
      {\large \institucion} \\
      \vspace{1mm}
      {\Large \titulo}
    \end{minipage} &
  \includegraphics[height=2.8cm]{dic.png}
  \end{tabular}
\end{center}

\vspace{4mm}
\begin{center}
  \includegraphics[width=0.65\textwidth]{placeholder.png}
\end{center}

\vspace{4mm}
\hrule
\vspace{1mm}
\hrule

\vspace{3mm}
\begin{tabular}{ll}
  Estudiante: & \estudiante\\
  Programa: & \programa\\
  Departamento: & \facultad\\
  Profesor: & \docente\\
  Fecha: & \fechaentrega\\
\end{tabular}

\vspace{3mm}
\hrule
\vspace{1mm}
\hrule

\clearpage
\tableofcontents

\clearpage

% ============================================================
% RESUMEN O INTRODUCCIÓN
% ============================================================
\section*{Resumen}
\textit{El presente proyecto desarrolla una aplicacion de navegacion interna que tiene como fin optimizar la orientacion y el desplazamiento dentro de la facultad de ingenieria de la universidad de magallanes. Esta solucion viene de la necesidad de mejorar la accebilidad y la eficiencia a la hora de encontrar salas, oficinas y laboratorios, especialmente para estudiantes nuevos y personas visitantes. Para el desarrollo se implemento un sistema de mapeo digital basados en mapas arquitectonicos del edificio, integrando tecnologias de posicionamiento basadas en puntos de referencia.}

\section{Introducción}
Los alumnos cuando llegan por primera vez a la universidad pueden sentirse desorientados a la hora de ir a alguna sala o laboratorio ya que no sabe como llegar a esos lugares en especifico, para resolver esta problemática se busca crear una aplicación móvil para poder navegar dentro de la facultad.

El proyecto fue desarrollado en el segundo semestre del año 2025 en las dependencias de la facultad de ingeniería de la universidad de Magallanes. Las pruebas y evaluaciones se realizaron usando planos digitales y simulaciones de desplazamiento en sus principales pasillos y áreas de uso común

\subsection{Objetivos}
Desarrollar una aplicación móvil de navegación interna que permita a los usuarios ubicarse y desplazarse eficientemente dentro de la Facultad de Ingeniería, utilizando tecnologías de posicionamiento y mapas digitales interactivos.
\begin{itemize}
  \item Analizar la estructura espacial y las necesidades de orientación dentro de la Facultad de Ingeniería..
  \item Diseñar e implementar una interfaz intuitiva que muestre mapas internos y rutas óptimas hacia los distintos destinos.
  \item Integrar tecnologías de posicionamiento en interiores (beacons, Wi-Fi o códigos QR) que permitan determinar la ubicación del usuario en tiempo real.
\end{itemize}

\section{Metodología}
Para la realización de este proyecto se siguieron las siguientes etapas:

\begin{enumerate}
  \item Digitalizacion de planos arquitectonicos de la facultad.
  \item Investigacion sobre la creacion de aplicaciones moviles.
  \item Preparación del entorno de desarrollo y repositorio de Github.
  \item Creación de la interfaz grafica de la aplicación.
  \item Integración de los mapas a la aplicación.
  \item Modelado del grafo de salas y conexiones.
  \item Implementacion del algoritmo de rutas.
  \item Implementación de funcionalidades extras a la aplicación.
  \item Pruebas piloto de la aplicación.
  \item Escalado a todos los pisos.
  \item 
\end{enumerate}

\section{Resultados Principales}
Incluya diagramas, tablas o fragmentos de código relevantes. Ejemplo de uso de figuras:

\begin{figure}[htbp]
  \centering
  \includegraphics[width=0.85\textwidth]{placeholder.png}
  \caption{Sustituir esta imagen por el diagrama o captura correspondiente}
\end{figure}

\section{Conclusiones}
Redacte conclusiones breves destacando los aprendizajes clave y posibles extensiones del trabajo.

\section*{Próximos Pasos}
\begin{itemize}
  \item Placeholder para acción futura 1.
  \item Placeholder para acción futura 2.
\end{itemize}

\section{Declaración de Uso de Apoyo}
  extit{Utilice esta sección para transparentar cualquier apoyo externo empleado durante la preparación del informe. Describa qué herramientas autorizadas se usaron, en qué etapa contribuyeron y cómo se integraron a su trabajo propio.}

\begin{itemize}
  \item Documente el uso de sistemas de digitalización (por ejemplo, OCR) indicando qué materiales se procesaron y cómo se verificó la fidelidad de la conversión.
  \item Declare el empleo de herramientas de IA generativa o de soporte al estudio (chatbots, asistentes de codificación, análisis de datos, etc.), especificando la finalidad, las indicaciones dadas y las verificaciones realizadas.
  \item Si el docente o la institución impone lineamientos adicionales, refleje aquí las condiciones y los permisos obtenidos.
\end{itemize}

\section{Referencias}
\begin{thebibliography}{9}
\bibitem{torvalds2001}
L. Torvalds y D. Diamond, \textit{Just for Fun: The Story of an Accidental Revolutionary}. HarperBusiness, 2001.

\bibitem{tanenbaum2015}
A. S. Tanenbaum y H. Bos, \textit{Modern Operating Systems}. 4.\,ed., Pearson, 2015.
\end{thebibliography}

\vspace{0.5cm}
\hrule
\begin{center}
\textbf{\estudiante}\\
Estudiante de \programa\\
\institucion\\
\fechaentrega{}
\end{center}

\end{document}
