\documentclass[a4paper,12pt]{article}
\usepackage[utf8]{inputenc}
\usepackage[spanish]{babel}
\usepackage{graphicx}
\usepackage{geometry}
\usepackage{tikz}
\usepackage{fancyhdr}

% Configuración de página
\geometry{
    a4paper,
    top=2cm,
    bottom=2cm,
    left=2cm,
    right=2cm
}

% Configuración de encabezado y pie de página
\pagestyle{fancy}
\fancyhf{}
\lhead{Sistema de Navegación UMAG}
\rhead{Código QR}
\cfoot{\thepage}

% Comando personalizado para crear una tarjeta de QR
\newcommand{\tarjetaqr}[3]{
    \begin{center}
        \begin{tikzpicture}
            \draw[line width=2pt, rounded corners=10pt] (0,0) rectangle (12,16);
            
            % Título superior
            \node[align=center, font=\Large\bfseries] at (6,14.5) {#1};
            
            % Espacio para el código QR
            \node at (6,8.5) {
                \includegraphics[width=7cm,height=7cm,keepaspectratio]{#2}
            };
            
            % Nombre del lugar
            \node[align=center, font=\LARGE\bfseries] at (6,3) {#3};
            
            % Instrucciones
            \node[align=center, font=\small, text width=10cm] at (6,1) {
                Escanea este código QR con la aplicación\\
                de navegación para iniciar tu ruta
            };
        \end{tikzpicture}
    \end{center}
    \newpage
}

\begin{document}

% ===============================================
% EJEMPLO DE USO:
% \tarjetaqr{TÍTULO}{ruta/al/codigo_qr.png}{NOMBRE DEL LUGAR}
% ===============================================

% Ejemplo 1: Escalera Centro Ciencias
\tarjetaqr{Sistema de Navegación UMAG}{piso1/P1_Escalera_Centro_ciencias.png}{Escalera Centro\\Ciencias - Piso 1}

% Ejemplo 2: Entrada Principal
\tarjetaqr{Sistema de Navegación UMAG}{piso1/P1_Entrada_1.png}{Entrada Principal\\Piso 1}

% Ejemplo 3: Patio de Ingeniería
\tarjetaqr{Sistema de Navegación UMAG}{piso1/P1_Patio_de_ingenieria.png}{Patio de Ingeniería\\Piso 1}

% ===============================================
% AGREGAR MÁS TARJETAS AQUÍ
% Simplemente copia y modifica la línea:
% \tarjetaqr{TÍTULO}{ruta_imagen.png}{NOMBRE}
% ===============================================

\end{document}
